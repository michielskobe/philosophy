\documentclass[../summary.tex]{subfiles}

\begin{document}
	
	\section{Introduction}
	We beginnen met een quote van Marshall McLuhan uit 1969. De quote vertelt ons dat de evolutie van technologie niet gestopt kan worden, en dat protest het proces niet zal stoppen. We zien een gelijkaardige statement, specifiek over AI, van Stuart Russell in 2021. Hij beweert dat superhuman intelligence er gegarandeerd aankomt, en dat het succes er van tot het einde van het menselijk ras kan leiden.  \\
	\\
	We focussen hier dus op de ontwikkeling van `general AI', komende van de `small AI' die we tegenwoordig utiliseren. Deze ontwikkeling is onvermijdelijk, en kunnen groei dus enkel proberen controleren om catastrofale gevolgen te voorkomen. \\
	\\	
	Deze les gaan we een diepere kijk nemen op de `Determination Thesis' (Thorstein Veblen 1900). Determinisme slaagt op het feit dat iets onvoorkombaar, noodzakelijk of causaal is. Deze thesis heeft verschillende invullingen van onvermijdelijke effecten zoals we verder in deze les zullen zien.  \\
	
	\subsection{Voorbeeld}
	Een eerste voorbeeld invulling is dat technologie altijd sociale effecten zal hebben:
	\begin{description}
		\item[pro:] Auto's, telefoons, drugs, social media, etc. zijn allemaal ontwikkelingen die drastische sociale veranderingen hebben meegebracht voor de maatschappij.
		\item[con:] De sociale effecten voorkomend uit technologie kunnen variëren. We spreken hier over de mensheid als een `open systeem', in wijze dat het onvoorspelbaar is tegenover een `gesloten systeem' dat altijd hetzelfde resultaat zou bieden.
		\item[con:]  De mens heeft in het verleden al veel voorspellingen gedaan, die compleet fout bleken te zijn. vb: Wright Brothers die dachten dat commerciële luchtvaart nog lang ging duren.
	\end{description}
	
	\subsection{Relevantie}
	Zoals eerder vermeld zijn er verschillende invullingen. Aangezien we industriële ingenieursstudenten zijn, gaan we deze les eerder focussen op technologisch determinisme. In dit techologisch determinisme zijn er vier types: Heidegger, social effects, creation of technology en evolution of technology. Bij de laatste twee is er een belangrijke morele en politieke relevantie.
	
	\subsubsection{Morele Relevantie}
	Als de ontwikkeling van een technologie gegarandeerd is, is diegene dat het uiteindelijk ontworpen heeft dan verantwoordelijk als de technologie wordt misbruikt? Men kan spreken van een zekere `responsibility gap'. \\
	\\
	Om dit idee van verantwoordelijkheid verder te verwerken, kijken we naar een gedachte experiment van Harry Frankfurt. Stel dat ik mijn vrouw wil vermoorden én dat ik een chip in mijzelf steek, mocht ik last-minute twijfels krijgen. In deze situatie zien we meteen dat de dood van zijn vrouw onvoorkombaar is, maar er zijn twee situaties:
	\begin{itemize}
		\item Ik krijg geen twijfels en dood mijn vrouw op eigen houtje. In dit geval is het duidelijk dat ik verantwoordelijk ben.
		\item Ik krijg last-minute twijfels en beslist mijn vrouw niet te vermoorden, waardoor de chip overneemt en mijn vrouw toch doodt. Ik kan dan niet meer verantwoordelijk genomen worden voor de dood.
	\end{itemize}
	Hieruit leren we dat, ondanks dat de gevolgen van iets vast liggen, er nog steeds een verschil kan zijn in verandwoordelijkheid. Een `responsibility gap' is in dit geval dus niet van toepassing. Met de focus op technologisch determinisme zal er dus iemand verantwoordelijk kunnen zijn, ondanks dat de creatie van een bepaalde technologie al vast ligt.
	
	\subsubsection{Politieke Relevantie}
	 De kwestie van verantwoordelijkheid kan meteen doorgetrokken worden naar een politiek vlak. We kunnen kijken naar het voorbeeld van Killer Robots.
	 % TODO kijk dees deel opnieuw in de les
	
	\section{First Determinism: Heidegger}
	Zoals in de introduction vermeld werd, zijn er verschillende soorten technologisch determinisme. Een eerste visie komt van Heidegger. Hij was een professor, maar ook nazi gericht. We kunnen dus in twijfel stellen of het gerechtvaardigd is om naar zijn visie te kijken. ZIjn filosofishe blik is echter niet per sé incorrect, door zijn politieke voorkeuren. ``Focus on the message, not the messenger''. \\
	\\
	Vooraleer we verder kijken naar Heidegger's mening, verklaren we eerst twee zaken die Heidegger veronderstelt:
	\begin{description}
		\item[Ontological Understanding: ] Hij veronderstelt dat elke mens een (spontaan, pre-reflexief, natuurlijk) begrip van de realiteit heeft. Men
		\item[Historicity: ] Hoe de werkelijkheid wordt verstaan/begrepen, is veranderd doorheen het verleden.
	\end{description}
	Heidegger stelt dat de huidige, moderne opvatting een instrumenteel karakter heeft. Dit betekent dat wij alles zien als een instrument of middel voor iets anders. vb. Data dient om een AI-systeem te trainen. Heidegger zegt dat we aan deze instrumentele blik vast hangen. Deze manier van denken kunnen we linken aan determinisme: wanneer we nadenken over iets, dan is dit gegarandeerd in een utilitaire of instrumentele blik.\\
	\\
	We moeten wel verduidelijken dat het in deze visie niet specifiek gaat over technologie als in concrete objecten, maar een technologische blik op de realiteit (instrumentele blik). Daarnaast stelt hij ook dat we net door de instrumentele blik meer technologieën zijn beginnen ontwerpen, en niet andersom. De instrumentele blik was er dus eerst. \\
	\\
	We argumenteren Heidegger's visie:
	\begin{description}
		\item[pro:] Veel hedendaagse voorbeelden: sport om fit te worden, educatie om een goede job te vinden, social media om af te spreken
		\item[con:] Het is een zeer sterke claim. Het is niet omdat er veel voorbeelden zijn, dat het overal en altijd onvermijdelijk is.
		\item[con:] Andere perspectieven zijn mogelijk. Vaak kan het lijken dat iemand instrumenteel over iets nadenkt, maar dat het eigenlijk een onbedoeld of toevallig effect is.
		\item[con:] De stelling is normatief omdat Heidegger doet blijken dat deze instrumentele blik gegarandeerd negatief is. Dit lijkt op het eerste zicht correct, bijvoorbeeld bij het instrumentaliseren van mensen en dieren. Het is echter pas problematisch als we louter over hen denken als instrument. De instrumentale blik is dus niet gegarandeerd negatief.
	\end{description}
	
	\section{Second Determinism: Social Effects}
	Een volgende invulling van de `Determination Thesis' gaat als volgt: technologie brengt onvermijdelijk sociale gevolgen met zich mee. Vooraleer we hierop ingaan bekijken we enkele veelvoorkomende misvattingen:
	\begin{itemize}
		\item Een bepaalde correlatie betekent niet meteen dat het causaal is.
		\item Als er geen bewijs is dat het bestaat, bewijst niet meteen dat het niet bestaat.
		\item Iets veroorzaakte niet per sé het ander, slechts omdat het eerste aan het andere voorafging.
	\end{itemize}
	We argumenteren deze invulling:
	\begin{description}
		\item[pro:] Ook hier zijn er veel voorbeelden van technologieën die wel degelijk een (drastische) sociale impact mee brachten. vb microgolf, auto, gsm, social media, etc.
		\item[con:] Het is niet omdat een sociaal gevolg wordt voorspeld, dat het ook een noodzakelijk, onvermijdelijk gevolg is. vb: een datacenter heeft bepaalde ecologische effecten, maar die gevolgen zijn niet altijd onvermijdelijk. 
	\end{description}
	We besluiten dat deze invulling van het technologisch determinisme niet correct is. Sociale effecten zijn niet deterministisch, maar eerder conditioneel. Een bepaalde technologie heeft pas sociale en niet-sociale effecten vanaf er aan bepaalde voorwaarden wordt voldaan. Zo is AI op zich niet problematisch, maar doordat er zoveel wordt in geïnvesteerd, ontstaan er meer drastische gevolgen. We opteren dus voor een `Conditional Thesis' in plaats van de `Determinism Thesis'.\\
	\\
	Stel dat technologie toch onvermijdelijk effecten heeft, dan zullen niet alle effecten sociale zijn. We kunnen besluiten dat ten eerste niet alle effecten onvermijdelijk zijn en ten tweede dat niet alle technologie sociale effecten zal hebben. We kunnen eerder zeggen dat technologie sociale effecten \emph{kan} sociale effecten hebben én sommige technologieën hebben onvermijdelijk effecten. 
	
	\section{Third Determinism: Creation}
	De volgende claim dat we zullen analyseren gaat als volgt: `Het ontwerp van technologieën liggen vast.' Protest voeren tegen het ontwerp van technologieën houdt geen steek, er zal altijd wel iemand zijn in de wereld die het zal ontwerpen.
	\begin{description}
		\item[pro:] Doorheen het verleden zijn er verschillende, mislukte pogingen geweest voor het ontwerp van een technologie tegen te gaan: vb. kruisboog, weefmachines, drukpers, kernbommen, etc.
		\begin{description}
			\item[con:] Protest heeft soms wel invloed. vb. kernbommen zijn gelimiteerd tot 9 landen.
		\end{description}
		\item[pro:] Het wiel werd zowel in Europa als in Amerika ontworpen. Echter in europa als hulpmiddel/transport en in Amerika als onderdeel voor speelgoed.
		\begin{description}
			\item[con:] Het wiel is in beide plaatsen ontwikkeld, maar speelgoed zien we niet als een noodzakelijk doel, dus is dit een argument tegen het determinisme. 
		\end{description}
		\item[pro:] \emph{Convergent Evolution} zegt dat in de biologie dezelfde veranderingen in verschillende organismen tegelijk opkomen, zonder dat zij iets met elkaar te maken hebben. vb. antifreeze in vissen aan de noord- en zuidpool. Deze convergente evolutie is ook te merken in de technologische wereld. vb. lamp en telefoon
		\begin{description}
			\item[con:] Niet elke technologie kende zulke convergente evolutie. vb. computer en vliegtuig
			\item[con:] Het is niet omdat een technologie tegelijk en onafhankelijk ontworpen werd, dat het ook noodzakelijk ontworpen werd. Het kan puur toeval zijn. Verder is iets ook niet per sé noodzakelijk, wanneer het niet toevallig is.
		\end{description}
	\end{description}
	We besluiten dat een convergente evolutie van technologie kan \emph{hinten} naar een noodzaak tot ontwerp van die technologie, maar het is geen concreet argument.\\
	\\
	De claim zelf kan nog steeds correct zijn, maar deze opgehaalde argumenten zijn niet voldoende om dit te bewijzen.
	
	\section{Fourth Determinism: Evolution}
	Over dit hoofdstuk worden geen vragen gesteld op het examen.
		
\end{document}
