\documentclass[../summary.tex]{subfiles}

\begin{document}
	
	\section{De natuur als grondstof}
	\subsection{Inleiding: "Grenzen aan de groei"}
	Het derde hoofdstuk van deze cursus behandelt de vraag of de natuur als grondstof gebruikt kan worden in het kader van duurzaamheid.  Als er  namelijk vragen gesteld worden bij duurzaamheid heeft het vaak te maken met het aanbotsen tegen de grenzen van wat het fysieke systeem aarde en nabije omgeving aankan. \\
	\\
	De vraag naar eventuele grenzen aan de groei is een relatief recente vraag.  De vraag werd voor het eerst op grote schaal gesteld in een in 1972 gepubliceerd boek  "The limits to growth". Deze publicatie werd samengesteld op vraag van de "Club van Rome", een groep van politici en wetenschappers. Er werden wiskundige modellen opgesteld om prognoses te maken over hoe ver we nog geraken met beschikbare grondstoffen en hoe de groei er in de toekomst uit zou kunnen zien. Deze methode van werken was nieuw in die tijd. De plotse wereldwijde bekendheid kwam voornamelijk doordat voor het eerst bleek dat de houdbaarheid van het menselijk leven niet kon gegarandeerd worden afhankelijk van de grondstof waarover we spreken. De cijfers die destijds gebruikt werden zijn intussen achterhaald en de prognoses kloppen uiteraard niet meer. Het idee dat de wereld iets begrensd is en de vraag of we op zo'n begrensde omgeving kunnen blijven groeien is wel bijgebleven. \\
	\\
	In de geschiedenis van het denken rond duurzaamheid is dit een belangrijke publicatie geweest, maar over het klimaat werd toen nog weinig gezegd.  Een andere belangrijke publicatie, gemaakt in de jaren 60 door Lynn White, behandelt de vraag hoe het mogelijk is geweest dat we in zo'n ecologische crisis gekomen zijn. In de publicatie maakt hij onder andere de conclusie dat de crisis voornamelijk tot stand is gekomen door de manier waarop de industrie ontwikkeld is in de Westerse wereld. Naar zijn analyse zou vooral de Christelijke traditie in het Westen daarmee te maken hebben. Voor de filosofie rond milieuaangelegenheden was dit een van de belangrijke mijlpalen in het zich ontwikkelende denken over het milieu en de verhouding tussen de mens en het milieu.\\
	\\
	In de komende sectie wordt dieper in gegaan op de redenen waarom de industriële revolutie, die een belangrijke oorsprong vormt voor de ecologische crisis, heeft plaats gevonden in West-Europa op die moment in de geschiedenis.
	\subsection{Culturele achtergronden voor industriële revolutie}
	In de Westerse cultuur waren drie belangrijke factoren aanwezig die kunnen gelinkt worden aan de Industriële revolutie, en niet of niet op dezelfde manier aanwezig waren in andere culturen.
	\subsubsection{Desacralisering van de natuur}
	Een eerste belangrijke factor die gelinkt kan worden aan de Industriële revolutie is de aanwezige desacralisering van de natuur. Dit betekent dat de natuur in deze cultuur niet als iets heilig aanzien wordt. De natuur is er, is omgeving die aanwezig is en waar we beroep kunnen op doen als het nodig is. In sommige culturen was of is de natuur nogsteeds sacraal, wat met zich meebrengt dat de natuur niet zomaar kan gemanipuleerd of gebruikt worden. \\
	\\
	De oorsprong van deze desacralisering van de natuur bevindt zich in het Christendom en de Griekse en Romeinse cultuur die in het Westen zijn doorgegeven op een bepaald moment. Een uitleg hiervoor is te vinden in het Jodendom, waaruit later het Christendom is ontstaan. Het Jodendom is ongeveer 3000 jaar geleden ontstaan uit nomadengroepen die na rondzwerven in het Midden-Oosten er zich stilaan gingen vestigen. Op die moment was er echter al een landbouwcultuur aanwezig in die omgeving die een specifieke verhouding hadden tegenover de natuur. Een landbouwcultuur is zeer sterk afhankelijk van de natuur dus dit werd als iets heilig beschouwd. Er waren Goden voor rivieren, de seizoenen, vruchtbaarheidsgoden, enzovoort. Bij een nomadenvolk was deze cultuur anders. Uiteraard zijn nomaden ook afhankelijk van de natuur en het weer, maar niet op dezelfde stabiele blijvende manier. Nomadengroepen trekken rond en zijn minder plaatsafhankelijk dan zo'n landbouwcultuur.\\
	\\
	Er is een interpretatie van het scheppingsverhaal dat in het Jodendom leefde op die moment dat God alles heeft geschapen, en dan aan hun gegeven om te beheren. De mens wordt hierin dus geplaatst als een soort heerser over de natuur en in de natuur zelf zit niets heilig meer. Dit stond in contrast met de landbouwcultuur waarbij voor onderdelen van de natuur verschillende Goden zijn, maar bevestigd wel het antropocentrisme in het Christendom.\\
	\\
	Ook andere Westerse culturen of godsdiensten zoals bijvoorbeeld de Islam kennen enkele verhalen die zeer gelijkaardig zijn aan Bijbelse scheppingsverhalen en dus ook dit antropocentrisme bevatten.\\
	\\
	De verschillende verhoudingen tussen mens, natuur en Goden kunnen gevisualiseerd worden in de "Metafysische driehoek".\\
	\\
	
	
	
\end{document}