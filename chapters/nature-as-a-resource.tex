\documentclass[../summary.tex]{subfiles}

\begin{document}
	
	\section{De natuur als grondstof}
	\subsection{Inleiding: "Grenzen aan de groei"}
	Het derde hoofdstuk van deze cursus behandelt de vraag of de natuur als grondstof gebruikt kan worden in het kader van duurzaamheid.  Als er  namelijk vragen gesteld worden bij duurzaamheid heeft het vaak te maken met het aanbotsen tegen de grenzen van wat het fysieke systeem aarde en nabije omgeving aankan. \\
	\\
	De vraag naar eventuele grenzen aan de groei is een relatief recente vraag.  De vraag werd voor het eerst op grote schaal gesteld in een in 1972 gepubliceerd boek  "The limits to growth". Deze publicatie werd samengesteld op vraag van de "Club van Rome", een groep van politici en wetenschappers. Er werden wiskundige modellen opgesteld om prognoses te maken over hoe ver we nog geraken met beschikbare grondstoffen en hoe de groei er in de toekomst uit zou kunnen zien. Deze methode van werken was nieuw in die tijd. De plotse wereldwijde bekendheid kwam voornamelijk doordat voor het eerst bleek dat de houdbaarheid van het menselijk leven niet kon gegarandeerd worden afhankelijk van de grondstof waarover we spreken. De cijfers die destijds gebruikt werden zijn intussen achterhaald en de prognoses kloppen uiteraard niet meer. Het idee dat de wereld iets begrensd is en de vraag of we op zo'n begrensde omgeving kunnen blijven groeien is wel bijgebleven. \\
	\\
	In de geschiedenis van het denken rond duurzaamheid is dit een belangrijke publicatie geweest, maar over het klimaat werd toen nog weinig gezegd.  Een andere belangrijke publicatie, gemaakt in de jaren 60 door Lynn White, behandelt de vraag hoe het mogelijk is geweest dat we in zo'n ecologische crisis gekomen zijn. In de publicatie maakt hij onder andere de conclusie dat de crisis voornamelijk tot stand is gekomen door de manier waarop de industrie ontwikkeld is in de Westerse wereld. Naar zijn analyse zou vooral de Christelijke traditie in het Westen daarmee te maken hebben. Voor de filosofie rond milieuaangelegenheden was dit een van de belangrijke mijlpalen in het zich ontwikkelende denken over het milieu en de verhouding tussen de mens en het milieu.\\
	\\
	In de komende sectie wordt dieper in gegaan op de redenen waarom de industriële revolutie, die een belangrijke oorsprong vormt voor de ecologische crisis, heeft plaats gevonden in West-Europa op die moment in de geschiedenis.
	\subsection{Culturele achtergronden voor industriële revolutie}
	In de Westerse cultuur waren drie belangrijke factoren aanwezig die kunnen gelinkt worden aan de Industriële revolutie, en niet of niet op dezelfde manier aanwezig waren in andere culturen.
	\subsubsection{Desacralisering van de natuur}
	Een eerste belangrijke factor die gelinkt kan worden aan de Industriële revolutie is de aanwezige desacralisering van de natuur. Dit betekent dat de natuur in deze cultuur niet als iets heilig aanzien wordt. De natuur is er, is omgeving die aanwezig is en waar we beroep kunnen op doen als het nodig is. In sommige culturen was of is de natuur nogsteeds sacraal, wat met zich meebrengt dat de natuur niet zomaar kan gemanipuleerd of gebruikt worden. \\
	\\
	De oorsprong van deze desacralisering van de natuur bevindt zich in het Christendom en de Griekse en Romeinse cultuur die in het Westen zijn doorgegeven op een bepaald moment. Een uitleg hiervoor is te vinden in het Jodendom, waaruit later het Christendom is ontstaan. Het Jodendom is ongeveer 3000 jaar geleden ontstaan uit nomadengroepen die na rondzwerven in het Midden-Oosten er zich stilaan gingen vestigen. Op die moment was er echter al een landbouwcultuur aanwezig in die omgeving die een specifieke verhouding hadden tegenover de natuur. Een landbouwcultuur is zeer sterk afhankelijk van de natuur dus dit werd als iets heilig beschouwd. Er waren Goden voor rivieren, de seizoenen, vruchtbaarheidsgoden, enzovoort. Bij een nomadenvolk was deze cultuur anders. Uiteraard zijn nomaden ook afhankelijk van de natuur en het weer, maar niet op dezelfde stabiele blijvende manier. Nomadengroepen trekken rond en zijn minder plaatsafhankelijk dan zo'n landbouwcultuur.\\
	\\
	Er is een interpretatie van het scheppingsverhaal dat in het Jodendom leefde op die moment dat God alles heeft geschapen, en dan aan hun gegeven om te beheren. De mens wordt hierin dus geplaatst als een soort heerser over de natuur en in de natuur zelf zit niets heilig meer. Dit stond in contrast met de landbouwcultuur waarbij voor onderdelen van de natuur verschillende Goden zijn, maar bevestigd wel het antropocentrisme in het Christendom.\\
	\\
	Ook andere Westerse culturen of godsdiensten zoals bijvoorbeeld de Islam kennen enkele verhalen die zeer gelijkaardig zijn aan Bijbelse scheppingsverhalen en dus ook  in bepaalde maten dit antropocentrisme bevatten. Hierbij is wel belangrijk om op te merken dat dit in de Islam eerder om een periode in de geschiedenis ging. Er is namelijk niet in de richting van een industrie-ontwikkeling doorgegroeid. Volgens sommige heeft dit te maken met het feit dat er op een bepaald moment een strekking in de Islam de bovenhand heeft gekregen die de mens weer terug op zijn plaats heeft gezet en de mens terug onderworpen aan God heeft geplaatst.\\
	\\
	De verschillende verhoudingen tussen mens, natuur en Goden kunnen gevisualiseerd worden in de "Metafysische driehoek".\\
	\\
	Een belangrijke toevoeging hieraan is het dualistisch wereldbeeld in de Griekse filosofie. Plato, een Griekse filosoof en schrijver, maakte namelijk een belangrijk onderscheid tussen een ideeënwereld en een materiële wereld. Een voorbeeld hiervan is "het konijn", iedereen weet hoe een konijn eruit ziet maar er bestaat niets als "het konijn" op zich want elk konijn is verschillend. Op die manier zei Plato dat theoretische concepten als "het konijn" niet bestaan op aarde maar op een aparte manier in een zogenaamde ideeënwereld. Op aarde komen we het voorbeeld van de konijnen dan tegen als pogingen om deze theoretische concepten of begrippen een concreet gestalte te geven.\\
	\\
	We krijgen dus een dualistisch mens- en wereldbeeld waarbij het materiële of het aardse minder belangrijk geacht wordt als het theoretische. Bij het ontwikkelende Christendom, dat zich over West-Europa aan het verspreiden was, gebruikten men elementen van die Platoonse filosofie en voedde dit alweer de gedachte dat het materiële minder belangrijk is dan het theoretische.\\
	\\
	Dit beeld zien we in de huidige samenleving ook nogsteeds in beperkte maten terug. Een voorbeeld hiervan is het onderscheid tussen arbeiders en bedienden: een arbeider werkt met zijn handen en is bezig met het materiële, maar wordt vaak minder betaald dan bedienden en is meestal ook niet "hoog opgeleid". Bedienden daarentegen zijn meer bezig met het theoretische en zijn vaak hoger geschoold met een hoger loon. Een tweede voorbeeld is de studiekeuze van een leerling die aan de middelbare school gaat beginnen. De gedachte leeft daarbij nogsteeds dat beginnen bij een richting met Latijn of wiskunde, het theoretische dus, belangrijk is en men later altijd kan afzakken naar iets lager en meer praktisch. Dit onderscheid tussen het theoretische en het materiële vloeit dus rechtstreeks voort uit de filosofie van Plato.\\
	\\
	Het Christendom is met andere woorden een combinatie van elementen afkomstig uit het Jodendom en het Hellenisme, de filosofie van Plato. We zien in de Westerse cultuur dus een dualistisch beeld tussen het spirituele en het materiële, waarbij het materiële vaak als minder belangrijk wordt beschouwd. Met zo'n beeld is het dus ook niet verwonderlijk dat de natuur in die cultuur niet als iets waardevol bekeken wordt, maar eerder iets dat aan ons ter beschikking gesteld wordt en we kunnen gebruiken als het ons goed uitkomt.\\
	\\
	\subsubsection{Visie op arbeid}
	In de Westerse cultuur heeft arbeid op een bepaald moment een positieve betekenis gekregen die niet of nauwelijks aanwezig is in andere culturen.\\
	\\
	Door opnieuw te kijken naar de geschiedenis van ideeën en mentaliteiten doorheen de verschillende voorafgaande culturen die zijn geëvolueerd in de Westerse cultuur valt op dat de Griekse filosofie niet aan de basis kan liggen. In de Griekse filosofie, zoals bijvoorbeeld bovenstaande filosofie van Plato, leeft het idee dat het theoretische belangrijker is dan het materiële en de mens zich dus met het spirituele moet bezighouden. In het Jodendom staat arbeid bekend als een "kwaad" afkomstig van Bijbelse verhalen waarin arbeid een straf betekende. In het Christendom was er op een bepaald moment een kantelpunt waarin arbeid een positievere betekenis kreeg. \\
	\\
	Het Christendom daarentegen had een kenmerkend eigenschap dat iedereen meetelt en belangrijk is, waarbij het jonge Christendom sympathie gevonden heeft bij groepen die zich aan de onderkant van de samenleving bevonden. Een voorbeeld hiervan is Jezus zelf, die de zoon was van een timmerman. Zijn apostelen waren zelf ook werkmensen zoals vissers, tentenmakers, enzovoort. Er zit dus duidelijk een trend in het jonge Christendom die gewone werkmensen respecteert en een andere manier om naar arbeid te kijken introduceert.\\
	\\
	Ook later in de geschiedenis, zoals bijvoorbeeld in de middeleeuwen, blijft deze positieve betekenis van arbeid in het Christendom duidelijk. De abdijen die op dat ogenblik overal over Europa te vinden waren bevatte duidelijk ook een component van werken: \textit{"ora et labora" ("bid en werk")}. \\
	\\
	De opkomst van het protestantisme in de 16e eeuw is een belangrijke mijlpaal in deze weg van de betekeniswijziging van arbeid in het Christendom. Er ontstond toen een plotse wending in de betekenis van arbeid waarbij werken voor het eerst echt iets positief werd. De mentaliteit veranderde naar een plicht waarbij iedereen hard moet werken zodanig dat God kan laten zien of Hij deze persoon goed of slechtgezind is. Als het harde werk  succesvol was, is het een teken dat God aan zijn/haar kant stond.\\
	\\
	Ook in deze tijd zijn er nog restanten te zien van deze plotse wending in het protestantisme. In landen als Nederland en de Scandinavische landen waar deze cultuur destijds populair was is het belang van werken en de economie nogsteeds duidelijk. Dit staat in contrast met landen en regio's waar het katholicisme populair was, zoals bijvoorbeeld Italië en Spanje, waar de nadruk op arbeid minder sterk is.\\
	\\
	Tot nu toe werd enkel de interpretatie gedaan vanuit de verschillende ideeën die geleden hebben tot het positieve beeld van arbeid. Het is ook mogelijk deze omslag te bekijken vanuit de geschiedenis van machtsverhoudingen, waarbij de uitbraak van de pest in Europa een belangrijk  kantelpunt was.\\
	\\
	\subsubsection{Vooruitgangsgedachte}
	Een laatste belangrijke factor om de industriële revolutie te verklaren is de manier waarop in de Westerse wereld gekeken wordt naar vooruitgang. Er leeft namelijk een zeer sterke gedachte van het constant vooruit gaan, "stilstaan is achteruit gaan", waaraan het idee van groeien ook gekoppeld kan worden.\\
	\\
	Een eerste belangrijke verklaring is het tijdsbeeld van het Westen. Het tijdsbeeld is een lineair tijdsbeeld, waarbij de geschiedenis voorgesteld wordt als een tijdsas met bepaalde tijdvakken. Hierdoor is dit ook de manier waarop we naar tijd kijken, als iets dat lineair vooruit gaat en we mee moeten volgen in die richting. Voor ons is dit vanzelfsprekend geworden, maar in andere culturen zijn of waren verschillende andere tijdsbeelden aanwezig. Een voorbeeld hiervan is de filosofie van Plato, waarbij er in de ideeënwereld geen verandering mocht zijn omdat de ideeën zo zuiver en volmaakt waren. Een verandering zou een evolutie in de graad van perfectie betekenen en dat mocht niet want alles was al perfect. Daarbovenop komt ook nog dat wanneer er geen verandering mogelijk is, er ook geen tijd mogelijk is, dus het tijdsbeeld was statisch in die filosofie. Andere mogelijkheden zijn een cyclisch tijdsbeeld in bijvoorbeeld landbouwculturen, waarbij perioden terugkomen in de toekomst in een bepaalde vaste cycli en het tijdsbeeld dus cyclisch is.\\
	\\
	Dit lineaire tijdsbeeld vindt zijn oorsprong opnieuw in het Jodendom, waarbij de nomadische cultuur een belangrijk aspect is. Nomaden trekken rond, komen ergens vandaag en gaan ergens naartoe. Zij zijn niet gebonden aan een cyclische cultuur zoals bijvoorbeeld de seizoenen in een landbouwcultuur. Het Christendom heeft dit tijdsbeeld later ook mee overgenomen en is nog meer toekomstgericht geworden.\\
	\\
	Het idee van vooruitgang komt dus rechtstreeks voort uit een lineair tijdsbeeld, andere tijdsbeelden laten evolutie of vooruitgang niet toe. \\
	\\
	Vooruitgang verklaren is echter niet vanzelfsprekend. We kunnen vooruitgang vergelijken met een punt van waar we een lijn trekken. De richting en zin van die lijn duiden dan aan wat vooruitgang is vanaf dat punt. Maar, uiteraard kunnen er pijlen in verschillende richtingen getrokken worden vanuit dat punt waardoor "vooruitgang" ineens iets anders betekent. Een voorbeeld hiervan zijn politieke partijen die elk met eigen meningen vooruitgang als iets anders bekijken.\\
	\\
	In de Westerse wereld hebben we de neiging om vooruitgang te koppelen aan techniek. Wanneer iets positief groeit, spreken we van vooruitgang. Ook in de economie is duidelijk dat groei belangrijk is en dat stilstaan achteruit gaan is. De vraag is natuurlijk of groei wel echt iets positief is, want zoals in het begin van dit hoofdstuk besproken zijn er wellicht ook grenzen aan groeien.\\
	\\
	In het kader van duurzaamheid wordt vaak met de I-PAT vergelijking gewerkt. Deze vergelijking geeft weer wat onze impact op de aarde bepaalt en ziet er als volgt uit:\\
	\begin{equation} \label{eqn}
		I = P * A * T
	\end{equation}
	I = impact\\
	P = population\\
	A = avarage level of wealth\\
	T = technology factor\\
	\\
	De impact (I) van de mens op de aarde, klimaat, milieu enzovoort wordt dus bepaald door de hoeveelheid mensen (P), de impact per persoon (A) die op zijn beurt kan gekoppeld worden aan de economie en rijkdom van die persoon, en de nodige middelen om die rijkdom/economie te genereren door middel van technologie (T). De impact betekent de groei, en als de groei te groot wordt moeten we de impact bepereken. Groei is namelijk niet altijd positief, we moeten kijken wat er groeit en te kosten van wat en daarmee tot een leefbare situatie te komen. Als we de impact onder controle willen houden kunnen we inzetten op een van die drie factoren.\\
	\\
	Een eerste mogelijkheid is om in te zetten op de populatie en deze dus te verminderen. Er blijkt dat de impact kleiner is als we onze rijkdom (A) moeten delen met minder mensen (P). Maar, een belangrijke eerste ethische kwestie is hoe en waar we de bevolking dan gaan inkrimpen. In de praktijk is dit dus moeilijk realiseerbaar\\
	\\
	 Er kan ook ingespeeld worden op de T, de impact per gegenereerde rijkdom. We zouden dus moeten efficiënter gaan produceren. Een voorbeeld hiervan is de gloeilamp die we hebben vervangen door efficiëntere verlichting. Maar, daarbij moeten we ook opletten op eventuele 'rebound'-effecten: we verlichten efficiënter, maar ook meer dan vroeger dus volgens de formule is de impact dan nogsteeds ongeveer even groot.\\
	 \\
	 Een laatste oplossing is om in te grijpen in de economie (A). We zouden namelijk kunnen in vraag stellen of de economie wel altijd moet groeien zoals economisten beweren. Als de economie zou krimpen zou onze impact op de aarde ook meteen kleiner worden maar dit inkrimpen roept veel weerstand op. Een mogelijkheid is om bepaalde rijkdom te verminderen, rijkdom die een grote impact heeft op de aarde maar weinig op welzijn. De economische groei vermeerdert de rijkdom van rijken, maar vanaf een bepaalde limiet zorgt groei niet voor méér geluk of welzijn. Welzijn is namelijk ook verbonden met openbare diensten zoals onderwijs, gezondheidszorg enzovoort. Het is beter als we zouden kunnen leven met een steady-state economie op een duurzaam niveau. Hiervoor zullen we wel moeten ontgroeien zodat we op een duurzaam niveau kunnen terechtkomen.\\
	 \\
	 \subsection{Besluit}
	 Een belangrijke take-away uit dit verhaal is hoe culturele en religieuze achtergronden onze houding beïnvloeden tegenover natuur, tijd en werk. Daarnaast is het ook belangrijk om na te denken hoe we ons positioneren in de discussies tussen "ontgroeien", "groene groei", ecomodernisten, techniekoptimisten, techniekpessimisten, enzovoort.
	
\end{document}