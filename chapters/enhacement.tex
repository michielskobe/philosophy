% !TeX spellcheck = nl
\documentclass[../summary.tex]{subfiles}

\begin{document}
	
	\section{Verbetering van het menselijk lichaam}
		In dit hoofdstuk nemen we een kijkje naar verbeterende technologie vanuit een ethisch perspectief. 
		\subsection{Inleidende opmerkingen}
			We hebben \textbf{veel te danken aan technologie}, veel moderne dingen (in vergelijking met 200 jaar geleden) hebben we als direct resultaat: politieke stabiliteit, democratie, pijn verzachting, hoge levensverwachting... . We kunnen echter niet vergeten dat deze technologie ook negatieve effecten heeft, daarom kunnen we niet zomaar \textbf{naïeve techno optimisten} zijn. \\
			\\
			In een ethische context kunnen we drie types verbeterende technologie voor het menselijk leven onderscheiden:
			\begin{description}
				\item[Genezen (therapie)] MRI om een ziekte te detecteren $\Rightarrow$ eliminatie van die ziekte $\Rightarrow$ gezond zijn
				\item[Preventief (therapie)] injectie tegen influenza $\Rightarrow$ preventie van een mogelijk toekomstige ziekte. 
				\item[Verbeteren (non-theraputic)] De verbetering van cognitieve en fysieke prestaties die verder gaan dan het `normale' of `natuurlijke' met behulp van (bio)technologie
				\begin{description}
					\item[Fysieke verbetering] Sneller en langer lopen, lastig fysiek werk
					\item[Cognitieve verbetering] Beer geheugen, intelligentie, langer studeren
					\begin{description}
						\item[Beta blockers] Harder, langer en beter werken
						\item[Nano technologie] Hersenweefsel verbinden met elektrische circuits 
						\item[Anabolische stereoiden] Gemakkelijker spierweefsel aanmaken
						\item[CRISPR] Perfecte kinderen kunnen maken %Ge kunt ook gewoon DJ De Lye zijn zaad gebruike 
					\end{description}
				\end{description}
			\end{description}
			Om een standpunt in te kunnen nemen over deze technologieën en hun aanvaardbaarheid moeten we eerst weten welke standpunten er zijn: 
			\begin{description}
				\item[Onaanvaardbaar] Het is \emph{niet} toe gelaten om deze technologie te gebruiken
				\item[Aanvaardbaar] Het is moreel oké om deze technologie te gebruiken. 
				\item[Verplicht] Men \emph{moet} deze technologie gebruiken $\Rightarrow$ het is moreel verplicht
				\item[supererogatoir] Het is goed om de technologie te gebruiken, maar niet verplicht
			\end{description} 
			We hebben ook argumenten nodig om deze standpunten te staven (reeds gezien in het ethiek vak uit het eerste jaar):
			\begin{description}
				\item[Deontologisch] `In principe' (in alle gevallen): het maakt niet uit wat de effecten zijn
				\item[Consequentie] Goede en slechte effecten die gekend en gewild zijn
				\item[Deugd ethiek] Expressie van morele eigenschappen
			\end{description}
			Er zijn ook twee soorten van positie innames:
			\begin{description}
				\item[Conservatief] Eerder tegen veranderingen: gebruikt vaak deontologische- en consequentie-argumenten.  
				\item[Liberaal] Eerder voor veranderingen: kiest vaak een `aanvaardbaar' standpunt. 
			\end{description}
			\mbox{}\\
			Wat kun je nu als filosoof doen met deze wetenshap: 
			\begin{itemize}
				\item Relevante concepten verklaren en uitwerken
				\item Detecteren en relativeren van gemaakte assumpties
				\item Argumenten onderzoeken en evalueren
				\item Redeneringen onderzoeken en evalueren
			\end{itemize}
		\subsection{General remarks}
		\subsection{Specific arguments}
		\subsection{Assumptions}
		
\end{document}