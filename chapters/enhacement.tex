% !TeX spellcheck = nl
\documentclass[../summary.tex]{subfiles}

\begin{document}
	
	\section{Verbetering van het menselijk lichaam}
		In dit hoofdstuk nemen we een kijkje naar verbeterende technologie vanuit een ethisch perspectief. 
		\subsection{Inleidende opmerkingen}
			We hebben \textbf{veel te danken aan technologie}, veel moderne dingen (in vergelijking met 200 jaar geleden) hebben we als direct resultaat: politieke stabiliteit, democratie, pijn verzachting, hoge levensverwachting... . We kunnen echter niet vergeten dat deze technologie ook negatieve effecten heeft, daarom kunnen we niet zomaar \textbf{naïeve techno optimisten} zijn. \\
			\\
			In een ethische context kunnen we drie types verbeterende technologie voor het menselijk leven onderscheiden:
			\begin{description}
				\item[Genezen (therapie)] MRI om een ziekte te detecteren $\Rightarrow$ eliminatie van die ziekte $\Rightarrow$ gezond zijn
				\item[Preventief (therapie)] injectie tegen influenza $\Rightarrow$ preventie van een mogelijk toekomstige ziekte. 
				\item[Verbeteren (non-theraputic)] De verbetering van cognitieve en fysieke prestaties die verder gaan dan het `normale' of `natuurlijke' met behulp van (bio)technologie
				\begin{description}
					\item[Fysieke verbetering] Sneller en langer lopen, lastig fysiek werk
					\item[Cognitieve verbetering] Beer geheugen, intelligentie, langer studeren
					\begin{description}
						\item[Beta blockers] Harder, langer en beter werken
						\item[Nano technologie] Hersenweefsel verbinden met elektrische circuits 
						\item[Anabolische stereoiden] Gemakkelijker spierweefsel aanmaken
						\item[CRISPR] Perfecte kinderen kunnen maken %Ge kunt ook gewoon DJ De Lye zijn zaad gebruike 
					\end{description}
				\end{description}
			\end{description}
			Om een standpunt in te kunnen nemen over deze technologieën en hun aanvaardbaarheid moeten we eerst weten welke standpunten er zijn: 
			\begin{description}
				\item[Onaanvaardbaar] Het is \emph{niet} toe gelaten om deze technologie te gebruiken
				\item[Aanvaardbaar] Het is moreel oké om deze technologie te gebruiken. 
				\item[Verplicht] Men \emph{moet} deze technologie gebruiken $\Rightarrow$ het is moreel verplicht
				\item[supererogatoir] Het is goed om de technologie te gebruiken, maar niet verplicht
			\end{description} 
			We hebben ook argumenten nodig om deze standpunten te staven (reeds gezien in het ethiek vak uit het eerste jaar):
			\begin{description}
				\item[Deontologisch] `In principe' (in alle gevallen): het maakt niet uit wat de effecten zijn
				\item[Consequentie] Goede en slechte effecten die gekend en gewild zijn
				\item[Deugd ethiek] Expressie van morele eigenschappen
			\end{description}
			Er zijn ook twee soorten van positie innames:
			\begin{description}
				\item[Conservatief] Eerder tegen veranderingen: gebruikt vaak deontologische- en consequentie-argumenten.  
				\item[Liberaal] Eerder voor veranderingen: kiest vaak een `aanvaardbaar' standpunt. 
			\end{description}
			\mbox{}\\
			Wat kun je nu als filosoof doen met deze wetenshap: 
			\begin{itemize}
				\item Relevante concepten verklaren en uitwerken
				\item Detecteren en relativeren van gemaakte assumpties
				\item Argumenten onderzoeken en evalueren
				\item Redeneringen onderzoeken en evalueren
			\end{itemize}
		\subsection{Algemene opmerking}
			We bekijken nu vier verschillende argumenten tegen cognitieve en fysieke verbetering
			\subsubsection{Het argument van intelligent ontwerp}
				Dit argument wordt vaak gebruikt in een religieuze context en gaat als volgt: (\textbf{P1}) de wereld bestaat uit een hiërarchie en (\textbf{P2}) het is moreel niet oké om deze rangorde te veranderen. Hieruit volgt dus dat (\textbf{C1}) verbetering onaanvaardbaar is. 
				\begin{description}
					\item[P1] Descriptieve stelling die beschrijft hoe de werkelijkheid in elkaar zit
					\item[P2] Evaluatieve of normatieve stelling die zegt dat het veranderen van god zijn intenties onrespectvol is tegenover hem
				\end{description}
				Er zijn enkele problemen met deze stellingen:
				\begin{description}
					\item[P1] (1) Er is geen enkele eigenschap die de verschillende lagen kan opdelen, (2) niet alle verbetering zal leiden tot een revolutie
					\item[P2] (1) Darwin suggereert dat de wereld geen product is van god, (2) misschien wilt god dat we verbeteringen maken
				\end{description}
			\subsubsection{Het argument van gelijkheid}
				Dit is een minder traditioneel maar meer hedendaags argument. (\textbf{P1}) Verbeteringen versterken of creëren ongelijkheden en (\textbf{P2}) deze ongelijkheid heeft negatieve gevolgen op onze samenleving. Hieruit volgt dus dat (\textbf{C1}) verbetering onaanvaardbaar is. 
				\begin{description}
					\item[P1] Dit is een empirische vaststelling die de wereld beschrijft en bevestigt of ontkracht kan worden
					\item[P2] Deze vaststelling ziet er in het algemeen juist uit. 
				\end{description}
				Er zijn enkele problemen met deze stellingen:
				\begin{description}
					\item[P1] (1) Er is geen deftige data omdat huidige technologie nog niet lang genoeg bestaat. (2) Sommige technologieën verminderen ongelijkheid (bvb. slaapproblemen oplossen).  (3) Goedkope technologie kan voor iedereen beschikbaar worden 
					\item[P2]  (1) Het salaris van een dokter is hoog omdat hij veel moeite moest doen om zijn positie te behalen.  (2) Het voordeel kan zo goed zijn dat het de nadelen overstijgt (bvb mensen die onderzoek doen naar kanker) 
					\item[C1] (1) Er kan een soort zwarte markt ontstaan voor middelen die verbeteringen beloven. (2) Als een ban tegenover deze middelen instelt, dan zal er wel een groep ontsnappen als uitzondering. (3) Bij een verbod moet er een inbraak op privacy gebeuren om te controleren op de naleving van het verbod.
				\end{description}
			\subsubsection{Het argument van gezondheid}
				(\textbf{P1}) Verbetering heeft een ongewilde impact op de gezondheid, hieruit volgt dat (\textbf{C1}) verbetering onaanvaardbaar is. 
				\begin{description}
					\item[P1] Dit is een empirische vaststelling die de wereld beschrijft en bevestigt of ontkracht kan worden. 
				\end{description}
				Er zijn enkele problemen met deze stellingen:
				\begin{description}
					\item[P1] Dit is niet waar voor alle verbeterende middelen
				\end{description} 
		\subsection{Specific arguments}
		\subsection{Assumptions}
		
\end{document}