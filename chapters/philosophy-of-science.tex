\documentclass[../summary.tex]{subfiles}

\begin{document}
	
	\section{Philosophy of science}
	
	\subsection{Role of human activity in climate change}
	
	We can ask ourself the question whether or not climate change is due to human activities. The answer to that question has changes over time due to a progressive understanding over the years as a result of climate research worldwide. Figure \ref{fig:progressive-understanding-climate-change} shows the evolution of this understanding. At this moment, it is unequivocal that human influence has warmed the atmosphere, ocean and land. 
	
	\begin{figure} [htbp]
		\centering
		\includegraphics[width=1\linewidth]{images/progressive-understanding-climate-change.png}
		\caption{Evolution of climate change understanding}
		\label{fig:progressive-understanding-climate-change}
	\end{figure}
	
	\subsection{The Precautionary Principle}
	
	The Rio Declaration of 1992 states that where there are threats of serious or irreversible damage, lack of full scientific certainty shall not be used as a reason for postponing cost-effective measures to prevent environmental degradation. This is known as the \textbf{Precautionary Principle} (\textit{het Voorzorgsprincipe}).
	\\\\
	Now what exactly is \textbf{full scientific certainty}? The ideal type of science is based on \textbf{reliable and rational knowledge} and not merely intuitive or emotional. Secondly, science should be \textbf{generalisable}. There can be no exceptions and it should be also valid for future, unknown cases. Space and time are uniform and science can not be merely anecdotic or casuistic. Furthermore, science should be \textbf{objective} and based on facts instead of interests. Science should also be \textbf{coherent} and can't be a loose set of statements. Lastly, there should be a \textbf{consensus in scientific community}. 
	\\\\
	Next, we will discuss what the demarcation criteria are to declare something as 'science'. The first criteria could be the use of numbers and mathematics. 
	
\end{document}