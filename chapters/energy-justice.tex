
\documentclass[../summary.tex]{subfiles}

\begin{document}
	
	\section{Energie rechtvaardigheid}
	\subsection{Energie transitie}
	
	De \textbf{energietransitie} heeft enerzijds een hele hoop \textbf{technische uitdagingen}. We moeten er bijvoorbeeld voor zorgen dat we elektrolyse kunnen doen op een manier die genoeg energie-efficiënt is. We hebben nucleair afval dat we ergens moeten opslaan. We maken gebruik van zonnepanelen, maar de zon schijnt niet altijd, dus we moeten het elektriciteitsnet balanceren aangezien de stroomproductie zeer variabel is. Dit zijn allemaal hele lastige technologische uitdagingen. \\
	\\
	Aan de andere kant zijn er ook \textbf{sociale uitdagingen} van de energietransitie. Een voorbeeld hiervan zijn de extreem hoge energieprijzen. In het geval van armoede, energiearmoede bijvoorbeeld. Energie-infrastructuur moet ook ergens staan en mensen hebben dat het liefst niet in hun buurt. Hierdoor ontstaan er protesten die dan meestal gaan over het feit dat mensen erop tegen zijn dat de energie-infrastructuur door hun achtertuin moet lopen. We hebben daarnaast ook allerlei materialen nodig voor het maken van windmolens, zonnepanelen enzovoort. Die materialen komen heel vaak van plaatsen en gemeenschappen waar arbeidsomstandigheden zeer slecht zijn. Tenslotte hebben we heel veel infrastructuur nodig die soms door natuurreservaten lopen om de energie bij de gebruikers te krijgen, maar dit is voor de natuur en de mensen daar niet geweldig. Dit soort problemen zijn minstens zo groot als de technologische aspecten. 
	
	
	
\end{document}